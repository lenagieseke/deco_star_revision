%!TEX root = ../deco_star.tex

%-------------------------------------------------------------------------

\section{Models}
\label{sec:models}

In the context of computer graphics, generation techniques are usually differentiated into procedural and data-driven approaches. This understanding applies equally to generating geometry, animations and texture. Procedural techniques describe the visual output by evaluating an algorithm, while data-driven approaches rely on existing data, such as photographs. However, as the field has developed, the approaches have begun to blend and their advantages have been combined. 
%through the continuous development of both fields, approaches started to blend and the advantages of both approaches are brought together.

This survey focuses on procedural models, but it also integrates and highlights promising or desirable characteristics of suitable data-driven techniques.

\subsection{Procedural}
\label{subsec:models_procedural}

\citeauthor*{ebert_2003_tmp}~\cite{ebert_2003_tmp} describe procedural techniques as algorithms and mathematical functions that synthesize a model or an effect. Representations based solely on equations are considered the ``purest'' form of procedural modeling~\cite{smelik_2014_aso}. This approach gained immediate importance in the early days of computer graphics. Analytical methods were able to reproduce natural phenomena such as wood, stone, water, smoke and plants using only a small amount of code, hence being memory efficient. The main appeal includes compactness combined with being continuous, scalable and not bound to a specific resolution.
% \mf{Remove the following sentence? It does not contribute much.}\legie{I think the sentence is a good reminder that e.g. modern HDMs such as the OcculusQuest have limited hardware and really need continous content and with that procedural models. In comparison, for e.g. VFX the mentioned characteristics are not really that relevant anymore and sometimes feel a bit old school.}\pa{I agree with Martin, not important, and not that true any more with increases in processing power.} 
\revremove{Condensing}{While procedural generation techniques have been a constant basis for generating content for games, their memory efficiency and unlimited resolution are more important than ever with the rise of, for example, virtual reality.}

The compactness and efficiency of a procedural model also enable parameterization, resulting in the model being responsive and flexible. Parameters usually represent certain visual characteristics and their prominence. Parameterization brings the crucial benefit of textures remaining editable throughout an entire visual effect production pipeline.

However, the effectiveness of traditional parameterization in helping an artist fulfill design goals is debatable. \citeauthor*{ebert_2003_tmp}~\cite{ebert_2003_tmp} argue that parameterization brings the benefit of a few parameters controlling many details. At the same time, this is potentially problematic for the realization of specific designs because these often require full individual control of all visual elements. Additionally, parameters are often non-intuitive due to representing overly abstract characteristics of the underlying functions and having overlapping effects~\cite{bourque_2004_ptm,lagae_2010_pis,gilet_2010_ias,benes_2011_gpm,lasram_2012_ssf,lasram_2012_ptp}.

In addition to difficulties in controlling a procedural representation, creating the  procedural model itself requires considerable effort, even though it is only a one-time investment. \rev{Clarification}{They require translating a visual design into a technical representation and generalizing it.} For procedural textures specifically, handling anti-aliasing efficiently can also be challenging. \citeauthor*{ebert_2003_tmp}~\cite{ebert_2003_tmp} includes a valuable and in-detail survey of function-based design principles of procedural models with a focus on textures. 

Procedural models are not limited to purely function-based designs. For example, the pioneering work of \citeauthor*{Prusinkiewicz_2012_TAB}~\cite{Prusinkiewicz_2012_TAB} applies the grammar-based L-system to algorithmically model plant growth, an approach extensively investigated by the computer graphics community.

The classifications of core mechanisms for procedural generation of \citeauthor*{hendrikx_2013_pcg}~\cite{hendrikx_2013_pcg} in the context of games and the categorization of \citeauthor*{smelik_2014_aso}~\cite{smelik_2014_aso} for virtual worlds are equally appropriate in the context of creative pattern generation. In the following we briefly discuss core mechanisms in the context of creative pattern generation, based on the previously mentioned taxonomies (\cite{hendrikx_2013_pcg}, \cite{smelik_2014_aso}).

\paragraph*{Stochastic Models} generate models by using random values. They can either be used in their original form as procedural models or as noise basis functions. Visual features can be added by combining multiple layers of noise in different resolutions. Typical noise functions are lattice value noise, lattice gradient noise (\eg Perlin noise \cite{perlin_1985_ais}), sparse convolution noise, and spectral noise~\cite{ebert_2003_tmp,lagae_2010_sap}. 
In the context of creative pattern generation, stochastic models build a basis for many designs but their design spectrum and controllability are limited.

\paragraph*{Function-based Models} extend the class of stochastic models by layering and combining a variety of functions to form a visually complex pattern. Typical building blocks are periodic, spline, step, clamp and conditional functions~\cite{ebert_2003_tmp} and are the basis for regular patterns designs. 

% \mfcomment{Perlin's image synthesizer \cite{perlin_1985_ais} does some of that already, his hypertexture paper \cite{perlin_1989_hyp} quite explicitly so (albeit with a focus on implicit shapes in 3D)} LG: not sure what to do with this comment.

\paragraph*{Rule-based Models} are part of often quite complex generation systems that can be context-dependent and/or design-specific. Rule-based models are programs that relate to and partition the space to fill and follow propagation rules. The algorithmic core often handles proxy shapes, while for the result graphical elements, such as vector graphics, are mapped to the proxies. Rule-based procedural models are suitable for creative pattern generation and novel control mechanisms because their iterative generation logic is open and flexible~\cite{wong_1998_cgf, mech_2012_tdf}. They can implement many designs and include any elements. Moreover, within a suitable pipeline, they can potentially take global constraints into consideration and build structural hierarchies.

\paragraph*{Grammar-based Models}
form grammatically-correct sentences from individual words, based on a system of rules, and they are related to rule-based models in that the rules are using grammars or rewriting systems. 
Prominent techniques are L-systems and shape grammars. An emerging subgroup of grammar-based models incorporates \textit{probabilistic} inference into the derivation of correct sentences from a grammar. In recent years, there have been a variety of successful grammar-based approaches for certain aspects of creative pattern generation~\cite{benes_2011_gpm,talton_2011_mpm,ritchie_2015_cpm}. However, grammars are difficult to set up and to design~\cite{stava_2010_ipm}. Because the execution process is inherently hierarchical, grammar systems have difficulty supporting creative control from a global to local scale.


\paragraph*{Artificial Intelligence Models} represent approaches that go beyond the direct execution of specific rules. For example, they automatically optimize results based on fitness or error functions, or they apply planning steps. \citeauthor*{hendrikx_2013_pcg}~\cite{hendrikx_2013_pcg} group this class into genetic algorithms, neural networks, and constraint satisfaction and planning. In recent years, machine learning has been introduced into procedural content generation with the same impact as in all other computer science fields~\cite{summerville_2017_pcg}. The potential of machine learning techniques in regard to creative pattern generation and their control mechanisms seems almost limitless.



\subsection[Data-Driven]{Data-Driven Models}
\label{subsec:design_models_datadriven}

%In contrast to procedural techniques, data-driven methods can be used in two ways in the context of creative pattern generation. First, they describe the processing of limited data, such as the pixels from a photograph. Second, they refer to the output of a method, which is again limited, such as pixel data. 
We classify methods for creative pattern generation as data-driven when they extrapolate from limited data such as the pixels from an image.
Data-driven models traditionally do not include underlying design models, as procedural representations do. Consequently, data-driven approaches are flexible in terms of possible designs and can achieve photorealism by processing real images.

At the same time, images often include visual features such as illumination effects that are unwanted and difficult to remove. Moreover, further down a production pipeline, pixel data is usually no longer editable. Working with high-resolution images leads to high memory requirements, and without additional algorithms, data is fixed to a given resolution and scale.

% Does this section really fit here?
Addressing the issue of resolution in example-based synthesis is a well-established field of research that aims to create an infinite amount of pixel data based on a given exemplar. The pyramid-based texture synthesis of \citeauthor*{heeger_1995_pbt}~\cite{heeger_1995_pbt} is an early famous example.~\citeauthor*{wei_2009_seb}~\cite{wei_2009_seb}, as well as~\citeauthor*{barnes_2017_aso}~\cite{barnes_2017_aso} present comprehensive summaries of such example-based texture synthesis techniques, discussing statistical feature matching, neighborhood matching, patch-based and optimization methods. Overall, example-based methods for texture synthesis have achieved similar results in data size, random accessibility and editing and resolution options as procedural textures---but only within specialized contexts and not in a unified manner. Procedural textures offer these capabilities as inherent characteristics.

Data-driven models are numerous and diverse because they need no underlying procedural model. In the following survey of the state of the art for creative pattern generation, we include various data-driven techniques, however an overall classification is not the focus of this work. Relevant techniques include the tiling and distribution of elements and drawing and brush mechanisms.


\subsection{Specific Pattern Designs}
\label{subsec:specific_pattern_designs}

In this section we review models that output specific pattern designs, which could be the basis for creative pattern generation. The references in this section summarize work that solely focuses on the output, offering little control or support for a creative creation process. Work that produces similar aesthetics but also offers control mechanisms (\eg \cite{wong_1998_cgf,yu_2012_ans,zehnder_2016_dso}) is reviewed in~\Cref{sec:analysis}.


Work on the generation of pattern designs is spread over various research communities. For the development of models \citeauthor*{whitehead_2010_tpd}~\cite{whitehead_2010_tpd} differentiates between two motivations in the context of generating models for ornamentation in games. First, the author identifies the goal of reproducing existing patterns such as Islamic and Celtic designs. Such work is mainly found in the communities of mathematics and computer science. Second, \citeauthor*{whitehead_2010_tpd} identifies the goal of generating novel pattern designs, which is held mainly by algorithmic computer artists. Such designs are usually not executed in an academic context and, beyond the presentation of the results, are unfortunately not well documented. Only a few exceptions, such as the work of \citeauthor*{takayama_2016_med}~\cite{takayama_2016_med} in 3D-printed ornate shapes or \citeauthor*{alvarez_2019_ido}~\cite{alvarez_2019_ido} in randomized abstract texture designs give much information about their underlying algorithms.

For what \citeauthor*{whitehead_2010_tpd}~\cite{whitehead_2010_tpd} calls mathematical/scientific ornamentation in the context of games, the author describes tiling and symmetry as the most relevant rules. Combined with interlacing parts of the pattern while repeating and tiling elements, these principles are able to systematically describe Islamic~\cite{ostromoukhov_1998_mtc} and Celtic~\cite{cromwell_1993_ckm} patterns, moving towards the traditional designs given in \Cref{fig:islamic_celtic_ornament}.

%LG: Take this figure out?
\begin{figure}
\centering
    \includegraphics[width=0.9\columnwidth]{figures/islamic_celtic_ornament_01.png}
    \caption[Islamic and Celtic pattern designs]{\label{fig:islamic_celtic_ornament}Examples of traditional Islamic (left) and Celtic pattern designs, showing the complexity of possible pattern designs. Image sources: [12, 13].} 
    % \mfcomment{source links are verbatim numbers, not pointing to the bibliography.}
    % lg: same as with figure 1. otherwise the list for figure one become unbearable.
\end{figure}

The seminal work of \citeauthor*{kaplan_2004_isp}~\cite{kaplan_2004_isp} presents an algorithmic representation of Islamic star patterns, a topic of ongoing interest~\cite{khamjane_2018_giq}. Readers further interested in this line of work are referred to the extensive investigation of~\citeauthor*{kaplan_2002_cgg}~\cite{kaplan_2002_cgg}. \citeauthor*{etemad_2008_apf}~\cite{etemad_2008_apf} and~\citeauthor*{hamekasi_2012_dpf}~\cite{hamekasi_2012_dpf} also focus on Islamic flower patterns. Celtic designs were also successfully computed by~\citeauthor*{kaplan_2003_cgc}~\cite{kaplan_2003_cgc} and~\citeauthor{doyle_2013_ccc}~\cite{doyle_2013_ccc}.

In addition to Islamic and Celtic designs, a variety of other pattern designs have been algorithmically formalized, such as Gothic window tracery~\cite{havemann_2004_gpd}, M. C. Escher patterns~\cite{dunham_1981_crh,kaplan_2004_isp}, woodwork~\cite{gulati_2010_acp,gulati_2012_acp}, optical illusions~\cite{chi_2014_ois}, mazes~\cite{pedersen_2006_ola} and tile-based patterns~\cite{ouyang_2015_cat, gdawiec_2017_pga}.


% %-------------------------------------------------------------------------

\revremove{Condensing}{
    
% \subsection{Summary}
% \label{subsec:models_summary}

To produce designs that include not only repetitive structures but also visual hierarchies and highlights for example, most procedural models in their "pure" form are too limited in their design variability. Data-driven models are more flexible in both their design space and creative control mechanisms. However, data-driven models are tedious, not suitable for automatically executing certain design rules, and they are limited in the data they use. The following review of the state of the art gives a detailed comparison of the capabilities for creative pattern generation.}