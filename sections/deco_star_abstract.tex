%!TEX root = ../deco_star.tex


\begin{abstract}
% Alternative title: A Survey of Creative Pattern Generation \newline

% This STAR contributes with a classification and analysis of control mechanisms for 2D creative pattern generation from an artist's perspective and discusses the techniques' potential for a creative workflow.

We review recent methods in 2D creative pattern generation and their control mechanisms, focusing on procedural methods. The review is motivated by an artist's perspective and investigates interactive pattern generation as a complex design problem. While the repetitive nature of a pattern is well-matched with algorithmic creation and automation, an artist needs more flexible control mechanisms for adaptable and inventive designs. 
We organize state of the art on the pattern design features they enable, such as repetition, frames, curves, directionality, and single visual accents. Within those areas, we summarize and discuss the techniques' control mechanisms for enabling artist intent. 
The discussion is led by the questions of (1) how input is given by an artist, (2) what type of content an artist inputs, (3) where the input affects spatially on the canvas, and (4) when input can be given in the timeline of the creation process. We categorize the available control mechanisms on an algorithmic level and their input modes in exemplars, parameterization, handling, filling, guiding, and placing interactions. To better understand the potential of the current techniques for creative design and make such an investigation more manageable, we motivate our discussion with the creative means of navigation, transparency, variation, and stimulation. We conclude our review by identifying possible new directions that can inspire innovation for artist-centered creation processes and algorithms.


%-------------------------------------------------------------------------
%  ACM CCS 1998
%  (see https://www.acm.org/publications/computing-classification-system/1998)
% \begin{classification} % according to https://www.acm.org/publications/computing-classification-system/1998
% \CCScat{Computer Graphics}{I.3.3}{Picture/Image Generation}{Line and curve generation}
% \end{classification}
%-------------------------------------------------------------------------
%  ACM CCS 2012
% (see https://www.acm.org/publications/class-2012)
%The tool at \url{http://dl.acm.org/ccs.cfm} can be used to generate
% CCS codes.
%Example:
% \begin{CCSXML}
% <ccs2012>
% <concept>
% <concept_id>10010147.10010371.10010352.10010381</concept_id>
% <concept_desc>Computing methodologies~Collision detection</concept_desc>
% <concept_significance>300</concept_significance>
% </concept>
% <concept>
% <concept_id>10010583.10010588.10010559</concept_id>
% <concept_desc>Hardware~Sensors and actuators</concept_desc>
% <concept_significance>300</concept_significance>
% </concept>
% <concept>
% <concept_id>10010583.10010584.10010587</concept_id>
% <concept_desc>Hardware~PCB design and layout</concept_desc>
% <concept_significance>100</concept_significance>
% </concept>
% </ccs2012>
% \end{CCSXML}

% \ccsdesc[300]{Computing methodologies~Collision detection}
% \ccsdesc[300]{Hardware~Sensors and actuators}
% \ccsdesc[100]{Hardware~PCB design and layout}

\begin{CCSXML}
<ccs2012>
<concept>
<concept_id>10010147.10010371</concept_id>
<concept_desc>Computing methodologies~Computer graphics</concept_desc>
<concept_significance>500</concept_significance>
</concept>
<concept>
<concept_id>10003120.10003123.10010860</concept_id>
<concept_desc>Human-centered computing~Interaction design process and methods</concept_desc>
<concept_significance>500</concept_significance>
</concept>
<concept>
<concept_id>10003120.10003123.10011758</concept_id>
<concept_desc>Human-centered computing~Interaction design theory, concepts and paradigms</concept_desc>
<concept_significance>500</concept_significance>
</concept>
<concept>
<concept_id>10003120.10003123.10011760</concept_id>
<concept_desc>Human-centered computing~Systems and tools for interaction design</concept_desc>
<concept_significance>300</concept_significance>
</concept>
</ccs2012>
\end{CCSXML}

\ccsdesc[500]{Computing methodologies~Computer graphics}
\ccsdesc[500]{Human-centered computing~Interaction design process and methods}
\ccsdesc[500]{Human-centered computing~Interaction design theory, concepts and paradigms}
\ccsdesc[300]{Human-centered computing~Systems and tools for interaction design}


\printccsdesc   

\end{abstract}
