%!TEX root = ../deco_star.tex


\begin{abstract}

% This STAR contributes with a classification and analysis of control mechanisms for 2D creative pattern generation from an artist's perspective and discusses the techniques' potential for a creative workflow.

We review recent methods in 2D creative pattern generation and their control mechanisms, focusing on procedural methods. The review is motivated by an artist's perspective and investigates interactive pattern generation as a complex design problem. While the repetitive nature of patterns is well-suited to algorithmic creation and automation, an artist needs more flexible control mechanisms for adaptable and inventive designs. 
We organize the state of the art on the pattern design features they enable, such as repetition, frames, curves, directionality, and single visual accents. Within those areas, we summarize and discuss the techniques' control mechanisms for enabling artist intent. 
The discussion is led by the questions of how input is given by the artist, what type of content the artist inputs, where the input affects the canvas spatially, and when input can be given in the timeline of the creation process. We categorize the available control mechanisms on an algorithmic level and categorize their input modes based on exemplars, parameterization, handling, filling, guiding, and placing interactions. To better understand the potential of the current techniques for creative design and to make such an investigation more manageable, we motivate our discussion with how navigation, transparency, variation, and stimulation enable creativity. We conclude our review by identifying possible new directions that can inspire innovation for artist-centered creation processes and algorithms.


% \mf{Is the categorization below correct? We do not discuss a tool FOR interaction design, but survey different ways of interaction. Similarly, we do not talk about "interaction design" process and methods – I have met colleagues who are active in adjacent fields, and it sometimes seems to me that they live in a totally different world ;-) }\legie{I don't know. Those are the categories that I thought fit best. Please let me know which ones to cut or which others might be better.}\pa{They look fine to me.}

\begin{CCSXML}
<ccs2012>
<concept>
<concept_id>10010147.10010371</concept_id>
<concept_desc>Computing methodologies~Computer graphics</concept_desc>
<concept_significance>500</concept_significance>
</concept>
<concept>
<concept_id>10003120.10003123.10010860</concept_id>
<concept_desc>Human-centered computing~Interaction design process and methods</concept_desc>
<concept_significance>500</concept_significance>
</concept>
<concept>
<concept_id>10003120.10003123.10011758</concept_id>
<concept_desc>Human-centered computing~Interaction design theory, concepts and paradigms</concept_desc>
<concept_significance>500</concept_significance>
</concept>
<concept>
<concept_id>10003120.10003123.10011760</concept_id>
<concept_desc>Human-centered computing~Systems and tools for interaction design</concept_desc>
<concept_significance>300</concept_significance>
</concept>
</ccs2012>
\end{CCSXML}

\ccsdesc[500]{Computing methodologies~Computer graphics}
\ccsdesc[500]{Human-centered computing~Interaction design process and methods}
\ccsdesc[500]{Human-centered computing~Interaction design theory, concepts and paradigms}
\ccsdesc[300]{Human-centered computing~Systems and tools for interaction design}


\printccsdesc   

\end{abstract}
