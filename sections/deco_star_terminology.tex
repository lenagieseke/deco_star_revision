%!TEX root = ../deco_star.tex

%-------------------------------------------------------------------------
\section{Terminology}\label{terminology}
The following describes how we will be using terms relevant to our topic. 

\textbf{Pattern} is a generic term for any type of repeated, often regular, arrangement~\cite{oed_2017}.

% \textbf{Ornament} constitutes a specific type of decor adhering to design rules, such as order, hierarchical structures, space adaptation and visual contrast and accents (\Cref{subsubsec:ornamentation}).

\textbf{Artistic} refers to a task with an outcome that potentially has meaning and value beyond aesthetics and practicality. In addition to formal skills that depend on a given domain, an artistic task usually requires creative thinking as well as intuition, emotion and sensual considerations, for example.

\textbf{Creative} refers to a task that intentionally produces a novel, non-standard outcome, as further discussed in \Cref{sec:creativ_means}. 

\textbf{Design space} refers loosely to the range of visual results a technique can create. For example, Perlin noise~\cite{perlin_1985_ais} has a rather restricted design space of noise images, only differing in few characteristics. Drawing with a pen can result in many different designs, thus resulting in a larger design space.

\textbf{Expressiveness} refers to the size, the variability and the openness of a design space as in detail discussed in~\Cref{sec:creativ_means}. Expressiveness is commonly used in the context of creative controls, but often without a clear understanding of its meaning.

\textbf{Canvas} constitutes the area in which the output is generated, similar to a canvas in a painting context.

\textbf{Shape} refers to the external boundary or outline on the canvas or of an object without any restrictions on the form. 

\textbf{Procedural} refers to the production of output by running an algorithm, such as a rule-based system.

\textbf{Data-driven} is the production of output based on given data.

% \rev{}{A }A \textbf{Parameterized} system is based on an implicit equation. However, in regard to control mechanisms and for this work, it refers to a system that offers separated, individually controllable characteristics. Parameterization commonly does not imply a procedural representation but can be part of any technique, including data-driven ones.

\rev{Condensing}{A \textbf{parameterized} system offers separately controllable characteristics. It can be used with procedural pattern representations, or data-driven approaches.}

