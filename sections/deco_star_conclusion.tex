%!TEX root = ../deco_star.tex


%-------------------------------------------------------------------------
\section{Conclusion}
\label{sec:conclusion}

The overall challenge addressed in this survey is to show how to support artists in their work with meaningful control mechanisms. The investigation of controllability is put into the context of two-dimensional creative pattern designs. Procedural models and the computation of designs offer novel approaches to create content and provide benefits over traditional manual manufacturing. However, providing control mechanisms that are intuitive to use and allow individual designs is an ongoing research challenge. For more complete and meaningful solutions aspects of both data-driven and procedural techniques are needed and must be merged into a unified whole.  

The reviewed techniques could complement each other and we hope to have furthered the direction of bringing different approaches together and to have carefully analyzed and emphasized an artist-centered perspective. This is the basis for developing innovative tools that further the ability of artists to create and to creatively express themselves.


\setcounter{secnumdepth}{0} %% no numbering
\section{Image References}
\label{sec:image_references}

\footnotesize
\sloppy

\begin{enumerate}
    \item{Manuscripts and Archives Division, The New York Public Library. 1450 - 1475. Historiated initial and another coat of arms. http://digitalcollections.nypl.org/items/510d47da-e47a-a3d9-e040-e00a18064a99}
    \item{Owen Jones. 1867. \textit{Examples of Chinese ornament selected from objects in the South Kensington museum and other collections.} London: S. \& T. Gilbert. http://archive.org/details/examplesofchines00jone}
    \item{The Miriam and Ira D. Wallach Division of Art, Prints and Photographs, The New York Public Library. 1882. Valentine cards utilizing decorative design, depicting fowers, hearts, butterflies and a tree. https://digitalcollections.nypl.org/items/510d47db-bc92-a3d9-e040-e00a18064a99}
    \item{Spencer Collection, The New York Public Library. 1910. Front doubleur. http://digitalcollections.nypl.org/items/8a6be0f9-3d78-b15e-e040-e00a180602c7}
    \item {Agnieszka Murphy. 2018. Polish folk art. 123RF, https://de.123rf.com/lizenzfreie-bilder/29119380.html?
    \&sti=nmw3eri7lnbl7fxnhi|\&mediapopup=29119380}
    \item{The Miriam and Ira D. Wallach Division of Art, Prints and Photographs, The New York Public Library. 1877. Arabesques : mosquée cathédrale de Qous : typan et éu- coinçons en faïence (XVIe. siècle). https://digitalcollections.nypl.org/items/510d47d9-66dd-a3d9-e040-e00a18064a99}
    \item{William Morris. 1876. African Marigold Printed Textile. Planet Art CD of royalty-free PD images William Morris: Selected Works. https://commons.wikimedia.org/wiki/
    File:Morris\_African\_Marigold\_textile\_drawing\_1876.jpg}
    \item{Colourbox. 2011. Frame with roses, Vector.\linebreak https://www.colourbox.com/vector/frame-with-roses-vector-1286656}
    \item{Colourbox. 2013. Illustration of frame in Ukrainian folk style, Vector. https://www.colourbox.com/vector/frame-vector-6826661}
    \item{Izabela Rejke. 2011. Traditional Polish Folk Design. http://rejke.deviantart.com/art/Traditional-Polish-Folk-Design-192417774}
    \item{Colourbox. 2013. Ornamental khokhloma oral postcard with seamless stripe, Vector. https://www.colourbox.com/vector/ornamental-khokhloma-oral-postcard-vector-8445572}
    
    \item{Free Patterns Area. Laser cut wood ornament template. 2018. https://www.freepatternsarea.com/designs/geometric-decorative-islamic-art-ornament-vector-design/. CC-BY-4.0 Creative Commons License}
    \item {Marcel's Kid Crafts. Celtic knot pattern. 2018. http://www.marcels-kid-crafts.com/celtic-knot-patterns.html. CC-BY-4.0 Creative Commons License.}
\end{enumerate}

% 